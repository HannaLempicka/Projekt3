\documentclass[10pt,a4paper]{article}

\usepackage{polski} 
\usepackage[utf8]{inputenc}
\usepackage[comma]{natbib}
\usepackage[top=2.5cm, bottom=2.5cm, left=2.5cm, right=2.5cm]{geometry}
\usepackage{amsmath,bm,amsfonts,amssymb}
\usepackage{wasysym}
\usepackage{xcolor}
\usepackage{graphicx}
\DeclareGraphicsExtensions{.pdf,.eps,.png,.jpg}
\usepackage{float}           
\usepackage{subfloat}          
\usepackage{caption}             
\usepackage{subcaption}
\usepackage{array}
\usepackage{multicol}   % \multicolumn{num_cols}{alignment}{contents}
\usepackage{multirow}   % \multirow{''num_rows''}{''width''}{''contents''} {* for the width means the content's natural width}
\usepackage{booktabs}           % tabele (toprule/midrule/bottomrule)
\usepackage{listings}
\usepackage{color} %red, green, blue, yellow, cyan, magenta, black, white
\definecolor{mygreen}{RGB}{28,172,0} % color values Red, Green, Blue
\definecolor{mylilas}{RGB}{170,55,241}

\newcommand{\logoGIK}{WGiK-znak.png}
\newcommand{\authorName}{Hanna Łempicka  \\ grupa 2a, Numer Indeksu: 291360}

\newcommand{\titeReport}{Projekt 3 - Kivy} % <<< insert short title in the food
\newcommand{\titleLecture}{Informatyka geodezyjna \\ sem. IV, ćwiczenia, rok akad. 2018-2019} % <<< insert title of presentation
\newcommand{\kind}{report}
\newcommand{\mymail}{\href{mail to:} {hania.lempicka@wp.pl}}
\newcommand{\supervisor}{....}
\newcommand{\gikweb}{\href{www.gik.pw.edu.pl}{www.gik.pw.edu.pl}}
\newcommand{\faculty}{Wydział Geodezji i Kartografii}
\newcommand{\university}{Politechnika Warszawska}
\newcommand{\city}{Warszawa}
\newcommand{\thisyear}{2019}
%\date{}
% PDF METADATA
\pdfinfo
{
	/Title       (GIK PW)
	/Creator     (TeX)
	/Author      (Hanna Łempicka)
}


% ------------------------- POCZATEK DOKUMENTU -------------------
\begin{document}
	% -----------------------------------------------------------------------------------------
	% ----------------------------  Title page
	% -----------------------------------------------------------------------------------------
	\begin{center} 
		\rule{\textwidth}{.5pt} \\
		\vspace{1.0cm}
		\includegraphics[width=.4\paperwidth]{\logoGIK}
		\vspace{0.5cm} \\
		\Large \textsc{\titeReport}
		\vspace{0.5cm} \\  
		\large \textsc{\titleLecture}
		\vspace{0.5cm}\\
		\textsc{\authorName}  \\
		\textsc{\faculty}, \textsc{\university}  \\ 
		\city, \thisyear
	\end{center} 
	\rule{\textwidth}{1.5pt}
	
	% -----------------------------------------------------------------------------------------
	% ----------------------------  Table of content
	% -----------------------------------------------------------------------------------------
	\tableofcontents 								% wyświetla spis treści
	%\addcontentsline{to}{chapter}{Spis treści} 	% dodaje pozycję do spisu treści
	%
	%

\newpage
\section{Cel ćwiczenia}
Zaprojektowanie aplikacji z wykorzystaniem frameworka Kivy, która będzie służyła do analizy ścieżek/tras zapisanych w formacje gpx. 

\section{Opis tworzenia aplikacji}
\subsection{Plik 'gpxDef.py'}
W tym pliku tworzyłam funkcje, użwyane potem w mojej aplikacji:
\begin{itemize}
	\item Funkcja 'czytanie':
	\\ Jej zmienną jest wczytywany plik. Wykorzystałam tu bibliotekę gpxpy do poprawnego odczytywania pliku typu gpx. Wyniki tej funckji są listy: lon (lista z długościami geograficznymi punktów), lat (lista z szerokościami geograficznymi punktów), el (lista z wysokościami puntów) oraz dates (lista z czasem na punkatch). Listy el i dates mogą być listami pustymi, gdy w wyjściowy plik nie zawiera danych na temat czasu i wysokości. 
	\item Funkcja 'Vincenty' 
	\\Zmiennymi tej funkcji są szerokości i długości geograficzne dwóch punktów. Pozwala ona na obliczenie odległości płaskiej między dwoma punktami.
	\item Funkcja 'dms':
	Jej zmienną jest godzina z jej częściami dziesiętnymi. Funkcja pozwala na zamienienie zmiennej na trzy czyli godziny, minuty i sekundy.
	\item Funkcja 'parametry':
	Jej zmiennymi są wyniki funkcji 'czytaj'. Liczę tu: odległość, prędkość średnią, przeywższenie całkowite, sumę podejść i zejść, wysokość maksymalną i minimalną, czas trwania treningu, przewyższenia na każdym odcinku, prędkość na każdym odcinku, odległość skośną, najszybciej/najwolniej przebyty odcinek oraz najtrudniejszy/najłatwiejszy odcinek. Za pomocą instrukcji warunkowej if pomijam w obliczeniach czas postoju. Dodatkowo uwzględniam przypadki gdy listy el i dates są puste (czyli jak pisałam wyżej, brak danych na ten temat). W tym wypadku dla wyników powiązanych z tymi listami przypisałam 'brak'.

\subsection{Plik 'mapview.kv'} 
W tym pliku tworzyłam wygląd mojej aplikacji. Pierwszą rzeczą było napisanie nazwy klasy do której odnosi się dana część wyglądu. Następnie definiowałam orientację aplikacji. Części aplikacji umieściłam w odpowiednich boxlayoutach, które utworzyły wiersze aplikacji. Używałam także button (łączyłam je za pomocą polecenia 'on press: root' z odpowiednimi funkcjami z pliku main.py), label (za ich pomocą stworzyłam opisy w aplikacji) oraz TextInput (ustawiałami ich id aby móc potem wykorzystać je w aplikacji, zmieniałam wielkości czcionki, wysokość oraz zablokowałam możliwość edycji). W LoadDialog wykorzystałam także FileChooserIconView (ustwiłam jego path).

\subsection{Plik 'main.py'}
Jest to plik główny mojej aplikacji. To tutaj tworzyłam funkcje do przycisków oraz wywołuję aplikację. Na początku zaimportowałam odpowiednie biblioteki. Następnie stworzył klasy.

\vspace{0.5cm}
\textbf{Klasa TrackForm:}
\begin{itemize}
	
	\item \textbf{Definicja init} 
	\\Zmieniłam źródło mapy oraz zadefiniowałam miejsce na wykres.
	
	\item \textbf{Klasa rysuj wykres:}
	\\Rysowanie wykresu przeywższenia od odległości i przewyższenia od czasu (jeśli są odpowiednie dane).
	
	\item \textbf{Open file:}
	\\Czyszczenie TextInput, mapy oraz wykresów oraz wczytywanie wybranego pliku. 
	
	\item \textbf{Analyse file:}
	\\Na początku sprawdzam czy został wczytany plik i czy ma dobre rozszerzenie. Jeśli nie, wyświetli się okienko z odpowiednią informacją. Następnie za pomocą funkcji z pliku 'gpxDef.py' odczytuję plik, liczę parametry dla niego oraz rysuję trasę na mapie. Następnie odpowiednie parametry przypisuję do opodwiednich TextInputów. Jeśli brakuje danych przypisuję informację o tym. \\Dodatkowo ustawiam zoom mapy oraz jest środek na miejsce odpowiadające trasie.
	
	\item \textbf{Draw rout:}
	\\Rysowanie ścieżki na mapie (funkcja wywoływana w analyse file)
	
	\item \textbf{Draw marker:}
	\\Rysowanie punktu na mapie (funkcja wywoływana w draw rout)
	
	\item \textbf{Load list, Dismiss Popup, Show Load:}
	\\Defincje pozwalające na wybór pliku z dysku (wywoływane w analyse file)
\end{itemize}

\vspace{0.5cm}
\textbf{Klasa Load Dialog:}
\\Klasa dla wyglądu wybierania pliku z dysku.

\vspace{0.5cm}
\textbf{Klasa MapViewApp:}
\\Połączeni z pliku kv z klasą TrackForm

\vspace{0.5cm}
Na końcu za pomocą polecenie run wywołuję aplikację.

\section{Link do repozytorium}
https://github.com/HannaLempicka/Projekt3.git

\end{document}